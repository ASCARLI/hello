\documentclass[UTF8]{ctexart}
\usepackage{geometry, CJKutf8, graphicx}
\geometry{margin=1.5cm, vmargin={0pt,1cm}}
\setlength{\topmargin}{-1cm}
\setlength{\paperheight}{29.7cm}
\setlength{\textheight}{25.3cm}

% useful packages.
\usepackage{amsfonts}
\usepackage{amsmath}
\usepackage{amssymb}
\usepackage{amsthm}
\usepackage{enumerate}
\usepackage{graphicx}
\usepackage{multicol}
\usepackage{fancyhdr}
\usepackage{layout}
\usepackage{listings}
\usepackage{float, caption}

\lstset{
    basicstyle=\ttfamily, basewidth=0.5em
}

% some common command
\newcommand{\dif}{\mathrm{d}}
\newcommand{\avg}[1]{\left\langle #1 \right\rangle}
\newcommand{\difFrac}[2]{\frac{\dif #1}{\dif #2}}
\newcommand{\pdfFrac}[2]{\frac{\partial #1}{\partial #2}}
\newcommand{\OFL}{\mathrm{OFL}}
\newcommand{\UFL}{\mathrm{UFL}}
\newcommand{\fl}{\mathrm{fl}}
\newcommand{\op}{\odot}
\newcommand{\Eabs}{E_{\mathrm{abs}}}
\newcommand{\Erel}{E_{\mathrm{rel}}}

\begin{document}

\pagestyle{fancy}
\fancyhead{}
\lhead{马竞轩, 3230103014}
\chead{数据结构与算法第六次作业}
\rhead{双十一, 2024}

\section{balance()函数}
\subsection{功能}
考察以t为root的子树,如果它不平衡,则判断它是不平衡中的哪种情况,通过旋转操作让它平衡。

注意!不管它是否平衡,都更新t的height,所以balance()是包含更新t的height这一操作的,下不赘述。
\subsection{参数}
指向一个子树root的指针,并且为引用,因此函数内部修改t时,直接修改实参。
\subsection{返回值}
无。
\subsection{功能实现}
考察它的左右孩子的height,若差值大于设定值ALLOWED\_IMBALANCE,再考察height大的那个孩子的左右孩子的height哪个大,据此分为四种情况,每种情况有对应的旋转方法,相对应做旋转就能使t为root的子树以及它的所有子树都平衡。

\section{我的remove()是如何实现的}
在remove()和detachMin()函数的最后一行加一行balance(),因为它们都存在删除操作,动态操作做完需要balance。

把detachMin()改为递归形式(上次作业是循环形式),这样才能形成调用堆栈,也就是从remove()中调用detachMin(t->right)开始,向下向左搜索最小值,找到后每向上回溯一次就检查balance一次,因为对下面节点进行balance操作也是动态操作,会影响上面节点的height。

remove()也是同理的,最下面remove完之后向上回溯检查balance。

具体来说,这两个函数的内容都是:

\{

        ...

        递归(即向下搜索)

        ...

        balance();

\}

也就是balance()在递归之后,所以是做完所有递归(搜索)再开始做balance(),也就是回溯的时候做balance(),要是balance在递归前,那就是反过来,搜索的时候做balance了,那就不对了。

\end{document}

%%% Local Variables: 
%%% mode: latex
%%% TeX-master: t
%%% End: 
