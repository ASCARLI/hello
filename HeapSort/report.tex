\documentclass[UTF8]{ctexart}
\usepackage{geometry, CJKutf8, graphicx}
\geometry{margin=1.5cm, vmargin={0pt,1cm}}
\setlength{\topmargin}{-1cm}
\setlength{\paperheight}{29.7cm}
\setlength{\textheight}{25.3cm}

% useful packages.
\usepackage{amsfonts}
\usepackage{amsmath}
\usepackage{amssymb}
\usepackage{amsthm}
\usepackage{enumerate}
\usepackage{graphicx}
\usepackage{multicol}
\usepackage{fancyhdr}
\usepackage{layout}
\usepackage{listings}
\usepackage{float, caption}

\lstset{
    basicstyle=\ttfamily, basewidth=0.5em
}

% some common command
\newcommand{\dif}{\mathrm{d}}
\newcommand{\avg}[1]{\left\langle #1 \right\rangle}
\newcommand{\difFrac}[2]{\frac{\dif #1}{\dif #2}}
\newcommand{\pdfFrac}[2]{\frac{\partial #1}{\partial #2}}
\newcommand{\OFL}{\mathrm{OFL}}
\newcommand{\UFL}{\mathrm{UFL}}
\newcommand{\fl}{\mathrm{fl}}
\newcommand{\op}{\odot}
\newcommand{\Eabs}{E_{\mathrm{abs}}}
\newcommand{\Erel}{E_{\mathrm{rel}}}

\begin{document}

\pagestyle{fancy}
\fancyhead{}
\lhead{马竞轩, 3230103014}
\chead{数据结构与算法第七次作业}
\rhead{Dec. 2nd, 2024}

\section{设计思路}
核心并不复杂,只有HeapSort一个函数。先创建空堆,此时数据已经存储在HeapSort中的vc(也就是test.cpp中的v),从第一个元素开始一个一个插入堆,而“插入”在代码中具体体现为“将push\_heap的两个参数中的后一个+1”,相当于文献中的vc.push\_back(),再进行push\_heap。所有数据遍历后,堆就创建完成了,默认为大根堆。

每次将最大值(也就是root)进行pop\_heap,这一函数具体做了“将root与堆的最后一个调换位置,此时的最后一个元素不算作属于堆的了,然后调整剩余元素使之仍为堆”,每次将pop\_heap的后一个参数-1,代表这个是已经排序好的,不再算作属于堆的了。由此从后往前排序。
\section{测试流程}
共4组,分别为随机序列、有序序列、逆序序列、部分元素重复序列,其中部分元素重复序列为生成0, 1, ..., int(size / 100) - 1如此循环的序列。每组先按要求生成数据存到vector v中,然后分别进行HeapSort、检查排序正确性、使用STL的HeapSort、清除数据。

\section{测试结果}
每次测试结果有误差(±30\%以内),这里取其中一组测试结果。

\begin{tabular}{|l|r|r|}
\hline
\textbf{} & \textbf{my heapsort time(ms)} & \textbf{std::sort\_heap time(ms)} \\ \hline
random sequence & 172 & 80 \\ \hline
ordered sequence & 92 & 88 \\ \hline
reverse sequence & 78 & 89 \\ \hline
repetitive sequence & 126 & 65 \\ \hline
\end{tabular}
    

\end{document}

%%% Local Variables: 
%%% mode: latex
%%% TeX-master: t
%%% End: 
