\documentclass[UTF8]{ctexart}
\usepackage{geometry, CJKutf8}
\geometry{margin=1.5cm, vmargin={0pt,1cm}}
\setlength{\topmargin}{-1cm}
\setlength{\paperheight}{29.7cm}
\setlength{\textheight}{25.3cm}

% useful packages.
\usepackage{amsfonts}
\usepackage{amsmath}
\usepackage{amssymb}
\usepackage{amsthm}
\usepackage{enumerate}
\usepackage{graphicx}
\usepackage{multicol}
\usepackage{fancyhdr}
\usepackage{layout}
\usepackage{listings}
\usepackage{float, caption}

\lstset{
    basicstyle=\ttfamily, basewidth=0.5em
}

% some common command
\newcommand{\dif}{\mathrm{d}}
\newcommand{\avg}[1]{\left\langle #1 \right\rangle}
\newcommand{\difFrac}[2]{\frac{\dif #1}{\dif #2}}
\newcommand{\pdfFrac}[2]{\frac{\partial #1}{\partial #2}}
\newcommand{\OFL}{\mathrm{OFL}}
\newcommand{\UFL}{\mathrm{UFL}}
\newcommand{\fl}{\mathrm{fl}}
\newcommand{\op}{\odot}
\newcommand{\Eabs}{E_{\mathrm{abs}}}
\newcommand{\Erel}{E_{\mathrm{rel}}}

\begin{document}

\pagestyle{fancy}
\fancyhead{}
\lhead{马竞轩, 3230103014}
\chead{数据结构与算法第四次作业}
\rhead{Oct.16th, 2024}

\section{测试程序的设计思路}

1. Create a new empty list called 'lst'.

2. Is 'lst' empty?

3. Push back 'lst' with even numbers(0~26). Copy 'lst' to 'lst2'. Pop front and pop back 'lst'.

4. Push front 1 in 'lst'.

5. lst2 = lst

6. Print the size of 'lst'.

7. use begin(), end() and '!=' in a loop used to iterate all the nodes in 'lst'.

8. Get front and back of 'lst'.

9. Find 6 and insert 5 before 6.

10. Delete 8 in 'lst'. Delete 12~16 in 'lst'.

11. Print the size of 'lst'.

12. Use erase() with 2 arguments to delete 2~2 in 'lst'.

13. Print the size of 'lst'.

14. 使用前置++,删除10后面那个元素

15. 使用后置++,删除5和5的next,并输出5之后两个节点的元素

16. Print 'lst' and 'lst2' before clear 'lst2'.

17. Is 'lst2' empty?

18. Clear 'lst2'.

19. Is 'lst2' empty?

20. Print the size of 'lst2'.

21. 用初始化列表的方法进行构造函数。

22. 用拷贝其他List方法进行构造函数。

23. 在main函数结束时调用析构函数。

\section{测试的结果}

测试结果一切正常。
运行结果:

\noindent\rule{\textwidth}{1pt}
\begin{lstlisting}
Create a new empty list called 'lst'.
lst:

Is 'lst' empty?
1

Push back 'lst' with even numbers(0~26). Copy 'lst' to 'lst2'. Pop front and pop back 'lst'.
lst:    2       4       6       8       10      12      14      16      18      20      22      24
lst2:   0       2       4       6       8       10      12      14      16      18      20      22      24      26

Push front 1 in 'lst'.
lst:    1       2       4       6       8       10      12      14      16      18      20      22      24
lst2:   0       2       4       6       8       10      12      14      16      18      20      22      24      26

lst2 = lst
lst:    1       2       4       6       8       10      12      14      16      18      20      22      24
lst2:   1       2       4       6       8       10      12      14      16      18      20      22      24

Print the size of 'lst'.
13

use begin(), end() and '!=' in a loop used to iterate all the nodes in 'lst'.
1       2       4       6       8       10      12      14      16      18      20      22      24


Get front and back of 'lst'.
1       24

Find 6 and insert 5 before 6.
lst:    1       2       4       5       6       8       10      12      14      16      18      20      22      24

Delete 8 in 'lst'. Delete 12~16 in 'lst'.
18
lst:    1       2       4       5       6       10      18      20      22      24

Print the size of 'lst'.
10

Use erase() with 2 arguments to delete 2~2 in 'lst'.
lst:    1       4       5       6       10      18      20      22      24

Print the size of 'lst'.
9

使用前置++,删除10后面那个元素
lst:    1       4       5       6       10      20      22      24

使用后置++,删除5和5的next,并输出5之后两个节点的元素
10

Before clear 'lst2'
lst:    1       4       10      20      22      24
lst2:   1       2       4       6       8       10      12      14      16      18      20      22      24

Is 'lst2' empty?
0

Clear 'lst2'.
lst:    1       4       10      20      22      24
lst2:

Is 'lst2' empty?
1

Print the size of 'lst2'.
0

用初始化列表的方法进行构造函数。
lst3:   1       2       3       4       5

用拷贝其他List方法进行构造函数。
lst:    1       4       10      20      22      24
lst4:   1       4       10      20      22      24

在main函数结束时调用析构函数。
\end{lstlisting}\noindent\rule{\textwidth}{1pt}



我用 valgrind 进行测试,发现没有发生内存泄露。

\noindent\rule{\textwidth}{1pt}
\begin{lstlisting}
All heap blocks were freed -- no leaks are possible
ERROR SUMMARY: 0 errors from 0 contexts (suppressed: 0 from 0)
\end{lstlisting}
\noindent\rule{\textwidth}{1pt}

\section{(可选)bug报告}

我发现了一个 bug,触发条件如下:

\begin{enumerate}
    \item main函数中,一旦一个名为name1的List被创建,那么它在main结束前永远无法全死掉,最多只能clear(),这不太符合常识(当然,我们不可能在main中使用析构函数)。比如说,我创建了一个名为temp的List,用完之后clear,然后又想用\lstinline|List(std::initializer_list<Object> il)|的方法初始化这个temp,但是这是redeclaration,编译器会报错。要么用循环push,要么换一个变量名,但是都不太舒服。
    注:因为复现bug会导致编译错误,所以我没有在main.cpp中复现这个bug。
    \item 其他bug可能算作异常处理,所以并没有详细阐述。例如,pop、erase时需要判断是否为空指针,find参数前后顺序相反时的处理等。
\end{enumerate}

\end{document}

%%% Local Variables: 
%%% mode: latex
%%% TeX-master: t
%%% End: 
